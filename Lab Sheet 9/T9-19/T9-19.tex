\documentclass{article}
\usepackage{amsmath}
\title{Basic LaTeX}
\begin{document}

\section{My first section}\label{first_section}

This is a section with a few subsections.

\subsection{A part of my first section}

Here I could write about the problem I'm trying to solve.

\subsection{Another part of my first section}

In this subsection I could solve the problem.

\subsubsection{Further fragmentation...}
anything really

\section{My second section}\label{second_section}

In Section \ref{first_section} we saw that...

A very helpful reference for LaTeX is \cite{Gratzer2007}.
Mathematics can be typed in to \LaTeX\ as $x^2$ and/or \((a+b)^2=a^2+2ab+b^2\).

\begin{equation}\label{my_first_equation}
e=mc^2
\end{equation}

In equation (\ref{my_first_equation}) we have a very well known relationship!

$$x^2 = 1 \text{ implies } x=\pm1$$

\begin{itemize}
    \item $a+b$
    \item $a-b$
    \item $-a$
    \item $ab$
    \item $a\cdot b$
    \item $a\times b$
    \item $a/b$
    \item ${a\over b}$
    \item $\frac{a}{b}$
\end{itemize}

$$\int_{0}^{\pi}4x^2\,dx$$

$$\begin{pmatrix}
a&b\\
c&d\\
e&f\\
\end{pmatrix}$$

$$\begin{matrix}
a&b\\
c&d\\
e&f\\
\end{matrix}$$

$$\begin{vmatrix}
a&b\\
c&d\\
e&f\\
\end{vmatrix}$$

\begin{align}
    (x+h)^2-x^2 & = x^2+2xh+h^2-x^2 && \text{(by distributivity)}\\
                & = 2xh+h^2         && \text{(by subtraction)}\\
                & = h(2x+h)         && \text{(by factorisation)}
\end{align}

$$
1+(-1)^n=\begin{cases}
            0, & \text{if $n$ odd}\\
            2, & \text{if $n$ even}
            \end{cases}
$$

\bibliographystyle{plain}
\bibliography{bibliography}
\end{document}
